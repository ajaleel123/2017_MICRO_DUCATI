%!TEX root=main.tex
\section{Summary}

\noindent Increasing application memory footprints and growing
thread-level parallelism on die has resurfaced TLB performance as an
important problem. This is because existing page walk caching
mechanisms are unable to accommodate the growing application memory
footprint. As a result, Last-Level TLB (LLT) misses suffer long
latency due to multiple memory accesses to the page table. When LLT
misses are frequent, applications desire low latency for address
translation.

This paper accelerates virtual to physical address translation by
improving on-die LLT coverage and LLT miss penalty. We improve on-die
TLB coverage by proposing {\em Unified Cache and TLB (UCAT)}, a
hardware mechanism that enables the conventional unified Last-Level
Cache (LLC) to also hold TLB entries. UCAT increases the on-die LLT
coverage by allowing as many TLB entries as there are cache lines in
the conventional on-chip LLC. We show that UCAT improves performance
by 60\% (up to 4x) on average across a set of memory intensive GPU
workloads. We show that these improvements can be realized with
negligible changes to the existing LLC architecture.

% This is because it still requires multiple requests to the page table
% for retrieving the address translation.

While UCAT improves performance, it does not decrease the number of
page table accesses on an LLT miss. To address this problem, we
propose {\em DRAM-TLB}, a hardware proposal that provides off-chip
support to extend the coverage of the on-chip TLB hierarchy. DRAM-TLB
serves as the next larger level in the TLB hierarchy architected in
DRAM technology and can be arbitrarily sized to provide the desired
TLB coverage. In steady state, DRAM-TLB can avoid multiple page table
accesses and provides address translation using a single stacked
memory access. Consequently, DRAM-TLB is a low latency alternative to
walking the page table on an LLT miss. Our studies show that DRAM-TLB
architected using emerging stacked memory technology improves
performance by 22\% on average (up to 2.25X). We show that DRAM-TLB
requires less than 1\% the capacity of our baseline 16GB stacked
memory system.

\aj{DUCATI summary}

% \newpage
% We have shown that emerging stacked memory technologies have now
% enabled a new TLB architecture previously was not practical with
% commodity DRAM. Since Stacked-TLBs enable high TLB coverage, we hope
% that they can spur future research directions in areas where
% extracting TLB performance has been challenging (e.g. virtualization).
% 
