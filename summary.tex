%!TEX root=main.tex
\section{Summary}

\noindent Increasing application memory footprints and growing
thread-level parallelism on die has resurfaced TLB performance as an
important problem. This is because existing page walk caching
mechanisms are unable to accommodate the growing application memory
footprint. As a result, Last-Level TLB (LLT) misses suffer long
latency due to multiple memory accesses to the page table. When LLT
misses are frequent, applications desire low latency for address
translation.

This paper accelerates address translations by improving on-die LLT
coverage and LLT miss penalty. We improve on-die TLB coverage by
proposing {\em Unified Cache and TLB (UCAT)}, a hardware mechanism
that enables the conventional unified Last-Level Cache (LLC) to also
hold TLB entries. UCAT increases the on-die LLT coverage by allowing
as many TLB entries as there are cache lines in the conventional
on-chip LLC. We show that UCAT improves performance by 60\% on average
across a set of memory intensive GPU workloads. We show that these
improvements can be realized with negligible changes to the existing
LLC architecture.

% This is because it still requires multiple requests to the page table
% for retrieving the address translation.

While UCAT improves performance, it does not decrease the number of
page table accesses on an LLT miss. Thus, we propose {\em
Stacked-TLB}, a hardware proposal that extends the coverage of the
on-chip TLB hierarchy. Stacked-TLB serves as the next larger level in
the TLB hierarchy and can be arbitrarily configured to increase TLB
coverage. In the common case, Stacked-TLB avoids multiple page table
accesses and provides address translation using a single stacked
memory access. Consequently, Stacked-TLB is a low latency high
bandwidth alternative to walking the page table on an LLT miss. Our
studies show that Stacked-TLB improves performance by 22\% on average
(up to 2X). We show that Stacked-TLB requires less than 1\% the
capacity of our baseline 16GB stacked memory system.

% \newpage
% We have shown that emerging stacked memory technologies have now
% enabled a new TLB architecture previously was not practical with
% commodity DRAM. Since Stacked-TLBs enable high TLB coverage, we hope
% that they can spur future research directions in areas where
% extracting TLB performance has been challenging (e.g. virtualization).
% 
