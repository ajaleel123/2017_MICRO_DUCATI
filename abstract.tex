%!TEX root=main.tex

\begin{abstract}

\noindent Conventional on-chip TLB hierarchies are unable to fully
cover growing application working-set. To make things worse,
Last-Level TLB (LLT) misses require multiple accesses to the page
table even with the use of page walk caches. Consequently, LLT misses
incur long address translation latency and hurt performance. This
paper proposes two low-overhead hardware mechanisms for reducing the
frequency and penalty of on-die LLT misses. The first, {\em Unified
CAche and TLB (UCAT)}, enables the conventional on-die Last-Level
Cache (LLC) to store cache lines and TLB entries in a single unified
structure and increases on-die TLB capacity significantly. The second,
{\em DRAM-TLB}, memoizes virtual to physical address translations in
DRAM and reduces LLT miss penalty when UCAT is unable to fully cover
total application working-set. DRAM-TLB serves as the next larger
level in the TLB hierarchy that significantly increases TLB coverage
relative to on-chip TLBs. The combination of these two mechanisms,
{\em DUCATI}, is an address translation architecture that improves GPU
performance by 81\% (up to 4.5x) while requiring minimal changes to
the existing system design. We show that DUCATI is within 20\%, 5\%,
and 2\% the performance of a perfect LLT system when using 4KB, 64KB,
and 2MB pages respectively.



%\noindent Conventional on-chip TLB hierarchies are unable to fully
%cover growing application memory footprints. To make things worse,
%Last-Level TLB (LLT) misses require multiple accesses to the page
%table (despite the use of page walk caches). Consequently, LLT misses
%incur long address translation latency. This paper focuses on reducing
%the frequency and penalty of on-die LLT misses. We propose {\em
%Unified CAche and TLB (UCAT)}, a hardware mechanism that enables the
%conventional on-die Last-Level Cache (LLC) to store cache lines and
%TLB entries in a single unified structure. Our evaluation using GPU
%workloads shows that UCAT increases on-die TLB capacity by 32x on
%average and improves GPU performance by 65\% on average (up to 4x).
%When UCAT is unable to fully cover total application memory footprint,
%we also propose {\em DRAM-TLB}, a hardware mechanism to memoize
%virtual to physical address translations in DRAM. DRAM-TLB serves as
%the next larger level in the TLB hierarchy that significantly
%increases TLB coverage relative to on-chip TLBs. We show that
%DRAM-TLBs architected using emerging stacked memory technology
%improves GPU performance by 22\% on average (up to 2.25x). Finally, we
%propose {\em DUCATI}, an address translation architecture that
%combines DRAM-TLBs and UCAT to reduce LLT miss penalty and improve
%on-die TLB coverage respectively. DUCATI improves performance by 81\%
%(up to 4.5x) while requiring minimal changes to the existing system
%design. We show that DUCATI is within 20\%, 5\%, and 2\% the
%performance of a perfect LLT system when using 4KB, 64KB, and 2MB
%pages respectively.

\end{abstract}



% Consequently, frequent misses in the Last-Level TLB (LLT) rely on the
% Memory Management Unit (MMU) for high performance virtual address
% translation.
% 


% LocalWords:  gigascale ARchitecture
