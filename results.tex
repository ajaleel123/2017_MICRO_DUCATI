% !TEX root = main.tex

% \begin {table*}[tp]
\small
\begin{center} 
\vspace{-0. in}
\caption{Summary of Distributed Placement and Stacked-TLBs for Different
Performance Metrics}
\vspace{-0. in}
\begin{tabular}{| c | c | c | c | c | }
\hline
Metric & Baseline & Distributed & Stacked-TLB & Perfect-LLT \\
       &          &  Placement  &             &             \\\hline
\# Page Table Lookups/LLT miss  & 1.75 & 1.75 & 1.00 &0.00 \\\hline
\multicolumn{5}{|c|}{ } \\ \hline
\multicolumn{5}{|c|}{ 4KB Page Size} \\ \hline
Throughput                            & 1.00 & 1.28 & 1.44 & 1.96 \\\hline
Relative LLT Miss Latency             & 1.00 & 0.70 & 0.51 & 0.00 \\\hline
Relative LLC Miss Latency             & 1.00 & 1.07 & 1.11 & 1.10 \\ \hline
\multicolumn{5}{|c|}{ } \\ \hline
\multicolumn{5}{|c|}{ 64KB Page Size} \\ \hline
Throughput                            & 1.00 & 1.04 & 1.12  & 1.34 \\\hline
Relative LLT Miss Latency             & 1.00 & 0.85 & 0.61  & 0.00 \\\hline
Relative LLC Miss Latency             & 1.00 & 1.03 & 1.05  & 1.07 \\ \hline

\end{tabular}
\label{table:results_summary}
\vspace{-0.015 in}
\end{center}
\normalsize
\end{table*}


\section{Results and Analysis} 
\label{sec:result}

\subsection{Results Summary}

\noindent Table~\ref{table:results_summary} summarizes several
performance metrics relative to the baseline system. Both Distributed
Placement and Stacked-TLB reduce LLT miss latency by 33\% and 49\%
respectively. The reduction in LLT miss latency improves performance
by 28\% and 44\% respectively. Stacked-TLB is best at reducing LLT
miss latency because the average page table accesses per LLT miss
reduce from 1.75 to a single Stacked-TLB access.

The table also shows that the different proposals tend to increase the
LLC miss latency by 7-10\%. This occurs because Distributed Placement
and Stacked-TLB both increase the traffic to stacked memory.
Consequently, the queuing delays in stacked memory increase. However,
the increase in LLC miss latency has minimal performance impact since
the large amount of memory-level parallelism present in these
workloads effectively hides the cache miss latency~\cite{bwa},

Figure~\ref{fig:summary_4k_pages} presents the per workload
performance behavior of Distributed Placement and Stacked-TLB relative
to a system with perfect LLT. The figure shows that distributed
placement bridges 30\% of the performance gap between the baseline
system and a perfect LLT while Stacked-TLB bridges 50\% of the
performance gap. In fact, for TLB-sensitive workloads like $CoMD$,
$BH$, $AMG\_2$, and $NEKBONE$ Stacked-TLB performs within 15\% of a
perfect LLT while using small page sizes.

\begin{figure}[tp] 
\vspace{-0 in} \centering
\centerline{\psfig{file=GRAPHS/4KB_pages_perf,angle=-90,width=\columnwidth}}

\caption{\small Performance with 4KB page size.\normalsize}
\label{fig:summary_4k_pages} 
\vspace{-0. in}
\end{figure}

\subsection{Sensitivity to Page Size}

\noindent We now present the performance of our schemes using 64KB
page size. For our 1024-entry baseline LLT size, using 64KB pages
increases the TLB coverage to 64MB (up to 256MB because of
~\cite{COLT}). In doing so, the LLT miss frequency reduces for
workloads that have high spatial locality in the virtual address
space. Table~\ref{table:results_summary} summarizes the different
performance metrics relative to the baseline system. With 64KB page
size, Distributed Placement and Stacked-TLB reduce LLT miss latency by
15\% and 39\% respectively. In turn, they improve performance by 4\%
and 12\% respectively.

With 64KB page size, Figure~\ref{fig:summary_64k_pages} presents the
per workload performance behavior of our baseline system, Distributed
Placement and Stacked-TLB relative to a perfect LLT system. We observe
that workloads with high spatial locality within a 64KB page tend to
observe fewer LLT misses. Consequently, the baseline system itself is
within 30\% of a perfect LLT (as opposed to 50\% for the 4KB page size
system). However, there is still opportunity to improve performance of
workloads that are TLB-sensitive with 64KB pages (e.g. $HPGMG$, $DMR$,
and $XSBENCH$). For these workloads, Distributed Placement and
Stacked-TLB proposals improve performance by 27-45\%. Overall,
Stacked-TLB bridges 30\% of the performance gap between the baseline
system and a perfect LLT.


% However, for workloads with limited
% spatial locality in a 64KB page (e.g. HPGMG, DMR) still benefit from
% our proposals.
% 
% We show that that distributed pagetable placement improves performance
% by 4\% while DRAM-TLBs and Stacked-TLBs improve performance by 14\%
% and 11\% respectively. We find the performance benefits from
% distributed page table placement diminishing with increasing page
% size. This is beacause fewer LLT misses cause the average number of
% memory requests to the memory system to increase. Since the majority
% of requests are serviced by the stacked memory, increasing queuing
% delays take away any benefit realized from page table placement. 
% 
% We note that it is exactly for this reason that DRAM-TLBs also
% outperform Stacked-TLBs at times. Since the majority of requests are
% serviced by stacked memory, average stacked memory access latency can
% be longer than the average system memory access latency (at least
% until system memory bandwidth saturates).

\begin{figure}[tp] 
\vspace{0. in} \centering
\centerline{\psfig{file=GRAPHS/64KB_pages_perf,angle=-90,width=\columnwidth}}

\caption{\small Performance with 64KB page size.\normalsize}

\label{fig:summary_64k_pages} 
\vspace{-0. in}
\end{figure}

\subsection{Sensitivity to Stacked DRAM Bandwidth}

\noindent Figure~\ref{fig:stack_bw_sense} illustrates the sensitivity
of our proposals to stacked memory bandwidth. We evaluate four
additional systems with 0.25X, 0.5X, 2X, and 4X the stacked memory
bandwidth of the baseline system (we do not vary the system memory
bandwidth). We increased the bandwidth by increasing the number of
channels in the stacked memory system. The y-axis illustrates the
average performance relative to the baseline system across all
workloads in the study.

The figure shows that when the stacked memory bandwidth is plentiful,
our proposals continue to improve performance. However, when the
stacked memory bandwidth becomes a bottleneck, the memory queuing
delays limit the performance of Distributed Placement. Under such
scenarios, Stacked-TLBs still improve performance since they reduce
the number of memory references. In general, our proposals efficiently
utilize the spare bandwidth available in the stacked memory system.
When the available bandwidth is high, performance improvements are
high. When the available bandwidth is low, performance improvements
are low.

\begin{figure}[t] 
\vspace{0.1 in} \centering
\centerline{\psfig{file=GRAPHS/sensitivity_stack_bw,angle=-90,width=\columnwidth}}

\caption{\small Sensitivity to stacked memory bandwidth. \normalsize}
\label{fig:stack_bw_sense} 
\vspace{-0 in}
\end{figure}

