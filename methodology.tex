% \begin{figure*}[tp] 
% \vspace{-0. in}
% \centering
% \centerline{\psfig{file=FIGURES/pagetable_placement,scale=0.80,angle=-90,width=\textwidth}}
% \caption{\small Different page table placement policies in a hybrid
% 	memory system. \normalsize}
% \label{fig:pagetable_placement} 
% \vspace{-0.0in}
% \end{figure*}

%\newpage
\section{Experimental Methodology}
\label{sec:method}

\noindent We assume a CPU-GPU heterogeneous system with a single CPU
and a single GPU supporting shared virtual memory~\cite{intelgen9,
amdzen} where the operating system and address translation services
are handled by the MMU and IOMMU respectively (see
Figure~\ref{fig:config}).

\begin {table}[h]
\begin{center} 
\vspace{-0.1in}
\caption{Baseline System Configuration}
\vspace{-0. in}
\begin{tabular}{|c|c|}
\hline
     \multicolumn{2}{|c|}{Streaming Multiprocessor (SM)}               \\ \hline
     Frequency            &  1 GHz                                    \\ 
     Number of SMs        &  128                                        \\ \hline
     \multicolumn{2}{|c|}{Baseline TLB and Cache Hierarchy}            \\ \hline
     L1 TLB   (private)   &  32-entry, 4-way                \\ 
     L2 TLB   (shared)    &  1024-entry, 8-way            \\ \hline
     L1 cache (private)   &  128KB,  4-way, 128B linesize  \\ 
     L2 Cache (shared)    &  8MB, 16-way, 128B linesize \\
     MSHRs for Mem Reqs   &  128 per memory channel  \\ \hline
     \multicolumn{2}{|c|}{Stacked Memory (HBM)}            \\ \hline
     Capacity             &  16GB (4 stacks)          \\
     Bus Frequency        &  1 GHz     \\ 
     Channels / Banks    &  32  / 16 per rank        \\
     Bus Width            &  128-bits per channel    \\ 
     Row Buffer Size      &  2048 Bytes              \\
     \small{tCAS-tRCD-tRP-tRAS}   &  14-14-14-33     \\ 
     Total Bandwidth      &  1024 GB/s              \\ \hline
     \multicolumn{2}{|c|}{Main Memory (DDR4)}               \\ \hline
     Capacity             &  256GB                   \\
     Bus Frequency        &  1600MHz (DDR 3.2GHz)    \\ 
     Channels / Banks    &  8  / 16 per rank        \\
     Bus Width            &  64 bits per channel     \\ 
     Row Buffer Size      &  2048 Bytes              \\
     \small{tCAS-tRCD-tRP-tRAS}   &  14-14-14-35     \\ 
     Total Bandwidth      &  204.8 GB/s               \\ \hline
     \multicolumn{2}{|c|}{Virtual Memory Configuration}     \\ \hline
%      ATS Lookup Latency   &  50 ns (oneway) \\
     Page Allocation      &  BW-aware Placement~\cite{batman,bwa}  \\ 
     Page Table (PT)      &  4-level hierarchical    \\ 
     Page Walk Caches     &  16 entries per PT level       \\ \hline

\end{tabular}
\label{table:method_system}
\vspace{-0.3in}
\end{center}
\normalsize
\end{table}


\subsection{System Configuration}

\noindent We use an industry proprietary event-driven performance
simulator to simulate a GPU (Table~\ref{table:method_system}) with
memory hierarchy that is loosely based on the NVIDIA Maxwell GPU
system~\cite{gpu_maxwell}. We model 128 Streaming Multiprocessors (SM)
that support 64 warps each. A simple round-robin warp scheduler
selects warp instructions each cycle. The baseline GPU memory system
consists of a two-level cache and TLB hierarchy. The first-level cache
and TLB are private to each SM while the last-level cache (LLC) and
last-level TLB (LLT) are shared by all the SMs. All caches use 128B
cache line size with 32B sectors. Our baseline also incorporates a
highly-threaded hardware page table walker for address
translation~\cite{power2014supporting, pichaigpu}.

% Each TLB supports compressing up to four contiguous TLB
% entries into a single entry~\cite{COLT}. 

We model a hybrid memory subsystem consisting of 16GB of stacked DRAM
(referred to as stacked memory) using High Bandwidth Memory (HBM)
technology~\cite{hbm-spec} and 256GB of commodity DRAM (referred to as
system memory) using conventional DDR4 technology~\cite{ddr4-spec}.
The stacked memory has 8X the bandwidth of system memory with similar
random access latency. The CPU and GPU have similar interconnect
latency to the system memory controller and stacked memory controller
(50ns in both cases). The memory controller supports 128-entry read
and write queues for each memory channel, open-page policy, minimalist
address mapping policy~\cite{minimalist} and FR-FCFS scheduling policy
(prioritizing reads over writes). Writes are issued in batches when
the write queue starts to fill up. 

% This isolates our performance studies to the bandwidth capability of
% the two memory systems rather than the latency to and from each
% individual memory system. We discuss performance implications with
% different interconnect latency (e.g. discrete CPU/GPU systems) in
% Section~\ref{sec:implications}.

We model a virtual memory system that maps virtual addresses to
physical addresses using random page replacement.  Our baseline
assumes 4KB page size, however we also study sensitivity to larger
page size (i.e. 64KB). To ensure full utilization of the total
available system bandwidth, physical pages are allocated based on the
bandwidth ratio of the hybrid memory system~\cite{bwa,batman}. In
doing so, both system memory and stacked memory satisfy memory
requests.

We model a four-level hierarchical page table~\cite{SkipPT}: Page Map
Level 4 (PML4), Page Directory Pointer (PDP), Page Directory (PD), and
Page Table (PT). All pages tables are initially allocated in low
bandwidth system memory in our baseline system. To speed up address
translation, we also model on-chip Page Walk Caches
(PWCs)~\cite{SkipPT, MMUcaches} for each page table level. The PWCs
are indexed by portions of the virtual address and are co-located with
the MMUs~\cite{MMUcaches}. We model a 16-entry PML4-cache, 16-entry
PDP-cache, a 16-entry PD-cache, and a 16-entry
PT-cache\cite{MMUcaches}. Note that PWCs tend to be relatively small
since they are power hungry, fully associative hardware structures.

\subsection{Workloads and Metric of Interest}

\noindent Our workloads consist of CUDA-based high performance
computing applications from the CORAL~\cite{CORAL},
Mantevo~\cite{mantevo}, and LoneStar~\cite{lonestar} suites (see
Table~\ref{table:bench_char}). The workloads are run with large inputs
to stress the hybrid memory system. We collected representative
program regions and warm up the caches, TLBs, and PWCs by
executing four billion warp instructions. After functional warmup, we
enable detailed timing simulation for two billion warp instructions.

Reduction in total execution time is our primary metric for
performance. We also report average TLB miss latency to correlate the
change in performance.

% collect average LLT miss latency. The reduction in LLT miss latency
% directly correlates with performance improvements, we
