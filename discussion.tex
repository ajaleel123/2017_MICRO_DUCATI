% \newpage
\section{DUCATI Discussion } 
\label{sec:implications}

% \subsection{Performance Impact On CPUs}
% 
% \noindent Our baseline studies assume on-die integrated CPU-GPU
% systems~\cite{intelgen9, amdzen} with a hybrid memory system. On such
% systems, both the GPU and CPU have similar access latency to both the
% stacked memory and system memory. While our have studies focused
% primarily on GPU workloads, both Page Table Placement and Stacked-TLB
% proposals are also expected to improve the performance of
% TLB-sensitive CPU workloads.
% 
% Our proposals also apply to discrete CPU-GPU systems where the CPU and
% GPU communicate using high speed links (e.g. QPI~\cite{intel_qpi},
% NVLink~\cite{bwa}). In commercially offered discrete systems
% today~\cite{hbm_intel, hbm_nvidia}, stacked memory is normally
% integrated on the GPU while system memory is normally integrated on
% the CPU. Thus, the GPU/CPU experiences longer unloaded access latency
% to the remote non-integrated memory (due to the additional round trip
% communication link latency).
% 
% Since the GPU is directly connected to the stacked memory on discrete
% CPU-GPU systems, the GPU continues to benefit from our stacked memory
% proposals to improve address translation. A natural question is how
% our proposals affect CPU performance on discrete CPU-GPU systems? At
% first, it may seem that our stacked memory proposals will degrade CPU
% performance because the unloaded access latency to stacked memory is
% longer than to the baseline system memory. However, CPU performance
% behavior depends on the bandwidth utilization and observed latency to
% system memory.
% 
% When system memory bandwidth is saturated, the stacked memory access
% latency tends to be lower than the observed system memory access
% latency. In fact, our studies revealed a 10\% lower stacked memory
% latency from the CPU relative to accessing the system memory. This is
% despite including the round trip communication link latency. As such,
% we expect our proposals to also improve CPU performance on discrete
% CPU-GPU system when the system memory bandwidth is saturated.
% 
% In situations where the system memory bandwidth is not saturated, we
% can ensure no impact on CPU performance, by always applying the
% baseline policy for CPU workloads. For example, the operating system
% can always follow System Memory Placement for all CPU workloads.
% Furthermore, the CPU can always follow the baseline policy of walking
% the page table when it misses in its own last level TLB, rather than
% probing the Stacked-TLB. A dynamic decision may be possible where the
% CPU can adapt to either using the baseline policy or our stacked
% memory proposals depending on the bandwidth utilization of system
% memory. We leave these optimizations for future work.
% 
% \subsection{Performance Impact on VM Operations}

\noindent Virtual memory operations such as remapping the virtual to
physical address or changes to page permissions require efficient
handling of page table update requests such as TLB shootdowns and TLB
flushes. It would be highly desirable that DUCATI does not impact the
overall latency of TLB shootdowns and TLB flushes.

TLB shootdowns normally occur as a part of an interrupt flow that
requires 5-30 microseconds ~\cite{zhengpagedgpu}. The access latency
for UCAT and DRAM-TLB tends to be a few ten to a few hundred nanoseconds. Thus, even in the worst case, the additional shootdown
latency is a small fraction of the overall interrupt latency (2-5\%).
As such, DUCATI has negligible performance impact on TLB shootdown
latency.

TLB flushes normally occur when the virtual to physical mappings of an
entire ASID need to be updated. Flushing conventional on-die TLB sizes
(e.g., 1024 entries in the baseline) can be done quickly. However,
flushing thousands of TLB entries (as in UCAT) or millions of TLB
entries (as in DRAM-TLB) can be a very long latency operation.
However, simple architecture support can be provided to flush large
TLBs without the long latency operation. For example, an {\em Epoch
Counter (EPCTR)} can be associated with each ASID. This counter tracks
the virtual to physical translations for an ASID in a given epoch. In
the DUCATI framework, the DRAM-TLB and UCAT can store the EPCTR in
addition to the physical address and page permission bits for each
entry.\ee{this is per 
entry in each of these structures? I've added ``for each entry'' above,
you can remove this comment if that's correct. The next sentence tells 
me it is correct, but just making sure.} When the
ASID requires a TLB flush, the EPCTR can simply be incremented. Thus,
DUCATI provides a valid translation only if the ASID and EPCTR stored
in the TLB-entry match the current EPCTR of the ASID. Both DRAM-TLB
and UCAT have sufficient storage space to support a 16-bit EPCTR per entry\ee{again added ``per entry''}. As
such, the TLB flush overhead can be completely eliminated with
negligible latency and hardware overhead.

% Though DRAM-TLB shootdowns and flush events consume bandwidth, these
% events occur infrequently and thus we expect minimal performance
% impact.

% TLB shootdowns can either be implemented either by performing a
% DRAM-TLB lookup and address comparison or simply by zeroing the
% DRAM-TLB-entry without reading its contents. On the other hand, the
% hardware overhead for flushing a DRAM-TLB is equivalent to
% initializing a large region of memory.
% 

